\documentclass{article}
\usepackage{feynmp}
\pagestyle{empty} %Takes away pagenumbers etc.
\begin{document}

\unitlength = 1mm
\begin{fmffile}{hggTT_diag} %Creating a diagram called ``my_process_diag''
\begin{fmfchar*}(80,50)\fmfpen{thick}%(x.y) sono la grandezza del diagramma
%Particelle stato iniziale(i) e finale (o)
\fmfleft{i1} 
\fmfright{o1,o2} \fmflabel{{\Large$\gamma$}}{o1} \fmflabel{{\Large$\gamma$}}{o2}
\fmfset{arrow_ang}{13}

\fmf{dashes,label={\Large$H$}}{i1,v1} 
\fmf{fermion,label={\Large $t$},label.side=right}{v2,v1}
\fmf{fermion,label={\Large $t$},label.side=right}{v1,v3}
\fmf{boson}{v2,o2}        
\fmf{boson}{v3,o1}        
\fmf{fermion,label={\Large $t$},label.side=right}{v3,v2}

\fmffreeze
\fmfforce{0.4w,0.5h}{v1}
\fmfforce{0.75w,0.75h}{v2}
\fmfforce{0.75w,0.25h}{v3}

%\fmfdot{v1,v2}% questo decide quali vertici abbiamo un punto marcato
\end{fmfchar*}
\end{fmffile}


%\fmf{plain,label=$\bar{T}$,label.side=left}{vx,v2} %questa linea in più serve per fare più lunga la freccia e piazzare il label
%%n.b: destra e sinistra sono stabiliti rispetto al senso della freccia
%\fmf{plain,label=$T$,label.side=left}{v2,vy} %questa linea in più serve per fare più lunga la freccia e piazzare il label
%\fmf{fermion}{vy,v4} %% questo Ú il T$

%infine piazzo le linee esterne
%\fmf{fermion}{o1,v3}%questo Ú anti top
%\fmf{dashes}{v3,o2}%questo Ú Higgs

%\fmf{fermion}{v4,o3}% questo Ú il top
%\fmf{dashes}{v4,o4}% questo Ú l'Higgs








\end{document}
