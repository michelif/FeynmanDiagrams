\documentclass{article}
\usepackage{feynmp}
\pagestyle{empty} %Takes away pagenumbers etc.
\begin{document}

\unitlength = 1mm
\begin{fmffile}{h_corr_diag} %Creating a diagram called "my_process_diag"
\begin{fmfchar*}(40,25)%(x.y) sono la grandezza del diagramma
%Particelle stato iniziale(i) e finale (o)
\fmfleft{i1} \fmflabel{$H$}{i1}
\fmfright{o1} \fmflabel{$H$}{o1}

\fmf{dashes}{i1,v1} 
\fmf{fermion,left,tension=0.2,tag=1,label=$t$,label.side=left}{v1,v2} 
\fmf{fermion,left,tension=0.2,tag=2,label=$t$,label.side=left}{v2,v1}
\fmf{dashes}{v2,o1}
\fmfposition


%\fmfdot{v1,v2}% questo decide quali vertici abbiamo un punto marcato
\end{fmfchar*}
\end{fmffile}


%\fmf{plain,label=$\bar{T}$,label.side=left}{vx,v2} %questa linea in più serve per fare più lunga la freccia e piazzare il label
%%n.b: destra e sinistra sono stabiliti rispetto al senso della freccia
%\fmf{plain,label=$T$,label.side=left}{v2,vy} %questa linea in più serve per fare più lunga la freccia e piazzare il label
%\fmf{fermion}{vy,v4} %% questo è il T$

%infine piazzo le linee esterne
%\fmf{fermion}{o1,v3}%questo è anti top
%\fmf{dashes}{v3,o2}%questo è Higgs

%\fmf{fermion}{v4,o3}% questo è il top
%\fmf{dashes}{v4,o4}% questo è l'Higgs








\end{document}