\documentclass{article}
\usepackage{feynmp}
\usepackage{color}
\pagestyle{empty} %Takes away pagenumbers etc.
\begin{document}

\unitlength = 1mm


\begin{fmffile}{VBF_diag}
\begin{fmfchar*}(90,50)\fmfpen{thick}%(x.y) sono la grandezza del diagramma
\unitlength = 1mm
\fmfset{arrow_ang}{13} %il default dello spessore delle frecce è 15 ma è troppo grosso

\fmfleft{i1,i2}
\fmfright{o1,o2,o3}
\fmf{phantom}{i1,v1,o1}
%\fmffreeze
\fmf{phantom}{i2,v2,o3}
%\fmffreeze
\fmfstraight
\fmf{phantom}{v1,v3,v2}
%\fmffreeze
\fmf{fermion, label={\Large$q$}}{i1,v1}
\fmf{fermion, label={\Large$q$}}{i2,v2}
\fmf{fermion, label={\Large$q^\prime$},label.side=left}{v1,o1}
\fmf{fermion, label={\Large$q^\prime$},label.side=right}{v2,o3}
\fmf{boson, label={\Large$W$}}{v1,v3,v2}
\fmffreeze
\fmf{scalar, label={\Large$H$}}{v3,o2}
%\fmffreeze%calcola posizioni dei vertici

\fmfforce{0.1w,.1h}{i1}
\fmfforce{0.1w,.9h}{i2}
\fmfforce{0.9w,.1h}{o1}
\fmfforce{0.9w,.9h}{o3}
\fmfforce{0.9w, 0.5h}{o2}

\end{fmfchar*}
\end{fmffile}

\end{document}
