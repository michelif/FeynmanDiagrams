\documentclass{article}
\usepackage{feynmp}
\pagestyle{empty} %Takes away pagenumbers etc.
\begin{document}

\unitlength = 1mm
\begin{fmffile}{prod_singola_diag}
\begin{fmfchar*}(40,25)%(x.y) sono la grandezza del diagramma
%Particelle stato iniziale(i) e finale (o)
\fmfleft{i1,i2} \fmflabel{$q$}{i2}\fmflabel{$g$}{i1}
\fmfright{o1,o2,o3} \fmflabel{$\bar{t}$}{o1} \fmflabel{$T$}{o2} \fmflabel{$q$}{o3} %la pecetta vecchia su o3 era q'

\fmf{gluon}{i1,v1}
\fmf{plain}{i2,v3,o3}
\fmf{boson, label=$Z$,label.side=left}{v3,v2} %la pecetta vecchia era W
\fmf{plain}{o2,v2}
\fmf{plain,label=$t$,label.side=left}{v2,v1}
\fmf{plain}{v1,o1}


%\fmfdot{v1,v2}% questo decide quali vertici abbiamo un punto marcato
\end{fmfchar*}
\end{fmffile}









\end{document}